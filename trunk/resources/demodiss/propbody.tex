
% Draft #1 (final?)

\vfil

\centerline{\Large Diploma in Computer Science Project Proposal}
\vspace{0.4in}
\centerline{\Large How to write a dissertation in \LaTeX\ }
\vspace{0.4in}
\centerline{\large M. Richards, St John's College}
\vspace{0.3in}
\centerline{\large Originator: Dr M. Richards}
\vspace{0.3in}
\centerline{\large 21 November 2000}

\vfil

\subsection*{Special Resources Required}
File space on Thor -- 25Mbytes\\
Account on the DEC Workstations -- 15Mbytes\\
An account on Ouse\\
The use of my own IBM PC (1000GHz Pentium, 200Mb RAM and 40Gb Disk).
\vspace{0.2in}

\noindent
{\bf Project Supervisor:} Dr M. Richards
\vspace{0.2in}

\noindent
{\bf Director of Studies:} Dr M. Richards
\vspace{0.2in}
\noindent
 
\noindent
{\bf Project Overseers:} Dr~F.~H.~King  \& Dr~S.~W.~Moore

\vfil
\pagebreak

% Main document

\section*{Introduction}

Many students write their CST and Diploma dissertations in \LaTeX\ and
spend a fair amount of time learning just how to do that. The purpos of 
this project is to write a demonsatration dissertation that explains in
detail how it done and how the result can be given to the Bookshop
on an MSDOS floppy disk for printing and binding.

\section*{Work that has to be done}

The project breaks down into the following main sections:-

\begin{enumerate}

\item The construction of a skeleton dissertation with the required 
structure. This involves writing the Makefile and makeing dummy files
for the title page, the proforma, chapters 1 to 5, the appendices and
the proposal.

\item Filling in the details required in the cover page and proforma.

\item Writing the contents of chapters 1 to 5, including examples
of common \LaTeX\ constructs.

\item Adding a example of how to use floating figures and encapsulated
postscript diagrams.

\end{enumerate}

\section*{Difficulties to Overcome}

The following main learning tasks will have to be undertaken before 
the project can be started:

\begin{itemize}

\item To learn \LaTeX\ and its use on Thor.

\item To discover how to incorporate encapsulated postscript into
a \LaTeX\ document, and to find a suitable drawing package on Thor
to recommend.

\item To discover what format the Bookshop would like for the finished
dissertation, and how to deal with postscript files that are too
large to fit on a single floppy disk.

\end{itemize}



\section*{Starting Point}

I have a reasonable working knowledge of \LaTeX\ and have convenient
access to Thor using an IBM PC in my office. Writing MSDOS disks is no 
problem.

\section*{Resources}

This project requires little file space so 25Mbytes of disk space on Thor
should be sufficient. I plan to use my own IBM PC to write floppy disks, 
but could use the PWF PCs if my own machine breaks down. 

Backup will be on floppy disks.

\section*{Work Plan}

Planned starting date is 01/12/2000.

\subsection*{Michaelmas Term} 

By the end of this term I intend to have completed the learning tasks 
outlined in the relevant section.


\subsection*{Lent Term}

By the division of term the overall structure of the dissertation
will have been written and tested.

By the end of term, example figures using encapsulated postscript
will have been included.
 

\subsection*{Easter Term}

On completion of the exams I will incorporate final details into 
the dissertation including a bibliography using bibtex and a table of contents.
The estimated completion date being 25/07/2001 to allow 
plenty of time should any unforeseen problems arise.

