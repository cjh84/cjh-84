\documentclass[12pt,a4paper,twoside]{article}
\usepackage{fancyhdr}

\setlength{\parindent}{0cm}
\setlength{\parskip}{2ex plus1ex minus 0.5ex}

\addtolength{\evensidemargin}{-2.5cm}
\addtolength{\oddsidemargin}{-0.5cm}
\addtolength{\textwidth}{3cm}

\addtolength{\headheight}{0.2cm}
\addtolength{\topmargin}{-1cm}
\addtolength{\textheight}{2.5cm}
% \addtolength{\footskip}{0.5cm}

\renewcommand{\_}{\texttt{\symbol{95}}}
\addtolength{\fboxsep}{0.1cm}
\newcommand{\param}[1]{\textit{\textrm{\textmd{#1}}}}
\newcommand{\codebar}{\rule{\textwidth}{0.3mm}}

\newcommand{\code}[1]
{\begin{center}\fbox{\parbox{16cm}{\texttt{#1}}}\end{center}}

\newcommand{\ret}[1]
{- \textrm{\textit{#1}}}

\fancyhead{}
\fancyhead[RO,LE]{\thepage}
\fancyhead[LO,RE]{SCOP Python Binding}
\fancyfoot{}
\pagestyle{fancy}

\newenvironment{bulletlist}
{
	\begin{itemize}
	\setlength{\itemsep}{0ex}
	\setlength{\parsep}{0ex}
}
{
	\end{itemize}
}

\newenvironment{alphalist}
{
	\begin{enumerate}
	\setlength{\itemsep}{0ex}
	\setlength{\parsep}{0ex}
	\renewcommand{\labelenumi}{(\alph{enumi})}
}
{
	\end{enumerate}
}

\newenvironment{numericlist}
{
	\begin{enumerate}
	\setlength{\itemsep}{0ex}
	\setlength{\parsep}{0ex}
}
{
	\end{enumerate}
}

\begin{document}

\centerline{\textbf{\LARGE SCOP Manual Annex --- Python Binding}}
\vspace{0.5cm}
\centerline{Version 1.4, 12th November 2003}
\centerline{David Ingram (\texttt{dmi1000@cam.ac.uk})}

{ \parskip 1mm plus 1pt \tableofcontents }

\section{Introduction}

This annex to the SCOP manual describes the Python language binding.
The binding implements the complete SCOP API, and is
itself written in Python.
The API is very similar to the C++ one, but
more concise because it is
simplified by Python's type system and garbage collector.

You should read at least the \textit{Programming model} section of the main
manual before this annex --- that information applies to
all language bindings and is not repeated here. The main manual
is also essential reading for discovering the precise semantics of
each command; here they are documented by example only.

\subsection{Using the library}

The library consists of three files in the \texttt{python/} directory
within the main distribution: \texttt{scop.py}, \texttt{scoplib.py} and
\texttt{scopxml.py}. You can install it by copying these to an
appropriate location for Python source on your system. Alternatively,
just set your \texttt{PYTHONPATH} environment variable to wherever you
have unpacked SCOP, for example:

\texttt{export PYTHONPATH=\$PYTHONPATH:\$HOME/scop/python}

To use the library from your Python programs, you only
need to reference the SCOP module, as follows:

\texttt{import scop}

This provides access to all the functions, including the level 2 XML
API, apart from the internal debugging routines.

\subsection{Example programs}

Python versions of the example programs are listed in this manual, and
can also be found in the directory \texttt{python/examples}. They
behave in an identical manner to their C++ and Java counterparts. You
can of course talk to the C++/Java versions of the servers with the
Python clients, and vice-versa.

There is one additional example, \texttt{status.py}, which simply shows how
to exercise the \texttt{list} function (the \texttt{scop} utility itself
serves this purpose for the C++ binding).

The \texttt{multiplex} wrapper for C's \texttt{select()} function
is not needed, because Python's \texttt{select()} native interface is
much cleaner than the C one.

\section{General semantics}

There is no need to free resources because buffers
are dealt with by the garbage collector. The connection itself
must be closed with the \texttt{scop\_close()} method, however.

Python exceptions are not thrown by the library (at least, not
intentionally) --- errors are generally indicated by returning None.

\subsection{Argument types}

The following arguments are all of type \textit{string}:
\texttt{remote\_hostname},
\texttt{name},
\texttt{endpoint},
\texttt{interest},
\texttt{message},
\texttt{method},
\texttt{text},
\texttt{request},
\texttt{reply}.

\texttt{log\_level} is an integer.
\texttt{unique} and \texttt{verify} are booleans represented
by integers.

\section{Level 1 API}

As with the other language bindings, the level 1 API provides
a single data type, namely strings.

The socket file descriptor used in the C++ binding is replaced by a
Python socket object (although in fact you don't need to know what type
it is since the \texttt{sock} object returned by \texttt{scop\_open()}
is simply passed back to the library so the other routines can identify
which connection to use).

\subsection{Connection establishment and teardown}

\code{%
scop\_open(remote\_hostname, name, unique = 0)
\ret{Returns socket object or None}\\
scop\_close(sock) \ret{No return value}\\
scop\_listen(sock, interest, unique = 0) \ret{No return value}
}

\subsection{Message functions}

\code{%
scop\_send\_message(sock, endpoint, message, verify = 0)\\
\ret{Returns an integer (-1 if not verifying)}\\
scop\_get\_message(sock)\\
\ret{Returns a tuple: (message string, rpc\_flag boolean), or None on error}
}
	
Note that \texttt{scop\_get\_message()} returns a tuple of which the
actual message is the first component (in the C++ API,
\texttt{rpc\_flag} is passed in as a pointer-to-int instead, and in
Java there is a separate \texttt{reply\_required()} method). This also
applies to the Level 2 \texttt{scop\_get\_struct()} described below.
Even if you don't intend to use the value of \texttt{rpc\_flag} don't
forget to assign the result to a tuple; not doing so is a common cause
of error.

\subsubsection*{Message passing example}

Right, that's enough calls for us to write our first example:

\codebar
\small
\begin{verbatim}
# sender.py - DMI - 7-11-03
# Usage: sender [ <message> ]   (default message is "Hello world!")

import scop, sys

if len(sys.argv) > 1:
   msg = sys.argv[1]
else:
   msg = "Hello world!"
sock = scop.scop_open("localhost", "sender")
if sock == None:
   print "Error on open"
   sys.exit()
scop.scop_send_message(sock, "receiver", msg)
scop.scop_close(sock)
\end{verbatim}
\normalsize
\codebar
\small
\begin{verbatim}
# receiver.py - DMI - 29-10-03

import scop, sys

sock = scop.scop_open("localhost", "receiver")
if sock == None:
   print "Error on open"
   sys.exit()
while 1:
   msg, rpc_flag = scop.scop_get_message(sock)
   print "Received <" + msg + ">"
   if msg == "quit":
      break
scop.scop_close(sock)
\end{verbatim}
\normalsize
\codebar

Since this was the first example, I was careful to check that
\texttt{scop\_open()} didn't return an error. To keep things concise
though, error returns from SCOP functions are not checked in the rest
of the examples, except for \texttt{sos.py}.

\subsection{Predefined event sources}

\code{%
scop\_set\_source\_hint(sock, endpoint)
\ret{No return value}\\
scop\_emit(sock, message, verify = 0)
\ret{Returns an integer (-1 if not verifying)}
}

\subsubsection*{Event sources example}

\codebar
\small
\begin{verbatim}
# event_source.py - DMI - 7-11-03
# Usage: event_source [ <source> ]   (default source is "news")

import scop, sys, time

count = 1
sock = scop.scop_open("localhost", "event_source")
if len(sys.argv) > 1:
   source = sys.argv[1]
else:
   source = "news"
scop.scop_set_source_hint(sock, source)
while 1:
   msg = "Item " + str(count)
   scop.scop_emit(sock, msg)
   count = count + 1
   time.sleep(1)
\end{verbatim}
\normalsize
\codebar
\small
\begin{verbatim}
# event_listener.py - DMI - 7-11-03

import scop, sys

sock = scop.scop_open("localhost", "event_listener")
scop.scop_listen(sock, "news")
while 1:
   msg, rpc_flag = scop.scop_get_message(sock)
   print "Received <" + msg + ">"
\end{verbatim}
\normalsize
\codebar
\small
\begin{verbatim}
# multi_listener.cpp - DMI - 7-11-03
# Usage: multi_listener [ <source-one> <source-two> ]
#    (default sources are "news" and "updates")

import scop, sys, select

sock = []
for i in range(2):
   sock.append(scop.scop_open("localhost", "multi_listener"))
   
if len(sys.argv) == 3:
   scop.scop_listen(sock[0], argv[1])
   scop.scop_listen(sock[1], argv[2])
else:
   scop.scop_listen(sock[0], "news")
   scop.scop_listen(sock[1], "updates")

while 1:
   read_fds = [sock[0], sock[1]]
   r, w, e = select.select(read_fds, [], [])
   for fd in r:
      msg, rpc_flag = scop.scop_get_message(fd)
      print "Received <" + msg + "> from ",
      if fd == sock[1]:
         print "updates"
      else:
         print "news"
\end{verbatim}
\normalsize
\codebar

\subsection{RPC functions}

\code{%
scop\_rpc(sock, endpoint, request, method = None)\\
\ret{Returns a string, or None on error}\\
scop\_get\_message(sock)\\
\ret{Returns a tuple: (message string, rpc\_flag boolean), or None on error}\\
scop\_send\_reply(sock, reply)
\ret{No return value}
}

The same \texttt{scop\_get\_message()} function is used to
get RPC requests and plain messages. If an RPC is expected
you should of course check that the second member of the tuple,
\texttt{rpc\_flag}, returns true.

\subsubsection*{RPC example}

\codebar
\small
\begin{verbatim}
# client.py - DMI - 8-11-03
# Usage: client [ <query> ]   (default query is "Hello world!")

import scop, sys

if len(sys.argv) > 1:
   query = sys.argv[1]
else:
   query = "Hello world!"
sock = scop.scop_open("localhost", "client")
reply = scop.scop_rpc(sock, "server", query)
print "Query <" + query + ">, Reply <" + reply + ">"
scop.scop_close(sock)
\end{verbatim}
\normalsize
\codebar
\small
\begin{verbatim}
# server.py - DMI - 8-11-03

import scop

sock = scop.scop_open("localhost", "server")
while 1:
   query, rpc_flag = scop.scop_get_message(sock)
   length = len(query)
   reply = ""
   for i in range(length):
      reply = reply + query[i] + query[i]
   scop.scop_send_reply(sock, reply)
scop.scop_close(sock)
\end{verbatim}
\normalsize
\codebar

\subsection{Admin functions}

\code{%
scop\_query(sock, endpoint)
\ret{Returns an integer}\\
scop\_clear(sock, endpoint)
\ret{No return value}\\
scop\_set\_log(sock, log\_level)
\ret{No return value}\\
scop\_terminate(sock)
\ret{No return value}\\
scop\_reconfigure(sock)
\ret{No return value}\\
scop\_list(sock)
\ret{Returns a list of (name, interest, src\_hint) tuples}
}

\subsubsection*{Status example}

This illustrates the use of the \texttt{scop\_list()} call.

\codebar
\small
\begin{verbatim}
# status.py - DMI - 8-11-03

import scop

sock = scop.scop_open("localhost", "status")
v = scop.scop_list(sock)
print len(v), "clients connected."
for tuple in v:
   name, interest, src_hint = tuple
   print "Client connection <" + name + "> listening to <" + interest + \
      ">, source hint <" + src_hint + ">"
scop.scop_close(sock)
\end{verbatim}
\normalsize
\codebar

\subsection{Cookies}

\code{%
scop\_set\_plain\_cookie(sock, text)
\ret{No return value}\\
scop\_get\_plain\_cookie(sock, name)
\ret{Returns a string}
}

\subsection*{Error-checking example}

Here's one final example, this time with proper error checking.
It's a server which simply logs all the messages it receives
using \texttt{syslog}.

\codebar
\small
\begin{verbatim}
# sos.py - DMI - 8-11-03

import scop, sys, syslog, os

sock = scop.scop_open("localhost", "sos", 1)
if sock == None:
   print "Can't connect to scopserver."
   sys.exit()

if os.fork() > 0:
   sys.exit() # Detach

while 1:
   buf, rpc_flag = scop.scop_get_message(sock)
   if buf == None:
      syslog.syslog(syslog.LOG_INFO, "Lost connection to scopserver.")
      sys.exit()
   syslog.syslog(syslog.LOG_INFO, buf)
scop.scop_close(sock)
\end{verbatim}
\normalsize
\codebar

\section{API Level 2 - Types with XML}

The Level 2 API adds support for non-string data types, including
structured data.

\subsection{General semantics}

The Python Level 2 API is completely different from the C++ and Java
versions, because it does not use any \texttt{pack()} or \texttt{extract()}
functions and there is no \texttt{vertex} data type. This is possible
because Python provides a list data type and the means to determine
the type of objects at runtime, which yields a radically simpler interface.

All you need to do is assemble your own data type using the Python
facilities for constructing lists and including any combination
of integers, floats and strings; then pass this object to SCOP.
The library will marshall it for you. Likewise, when receiving data
it is extracted from the XML automatically and handed to you
as the appropriate native Python object.

There is no explicit support for SCOP's binary types, but fortunately
this isn't necessary since Python strings can include NULLS. The
solution adopted by the library to handle strings is to check if it has
been passed entirely printable characters. If so it is packed as
an XML string, otherwise it is sent as binary (using the SCOP
protocol's hex encoding). The Python binding can also receive binary
types sent by programs written in other languages, but these are always
returned as Python strings.

\subsection{Argument types}

The following parameters are all \textit{polymorphic}:
\texttt{data},
\texttt{args},
\texttt{request},
\texttt{reply},
\texttt{v}.
In the restricted sense of SCOP this means they may either be
integers, floats, strings, or lists (containing the same types).
Some of the function return values are also polymorphic in the same sense.

Note that parameters named \texttt{request} and \texttt{reply} are
encoded without using XML if they happen to be strings (see the level 1
API).

\subsection{XML Message functions}

\code{%
scop\_send\_struct(sock, endpoint, args, method = None)
\ret{No return value}\\
scop\_get\_struct(sock)
\ret{Returns a tuple: (polymorphic, rpc\_flag boolean)}
}

\subsubsection*{XML example}

This address book example is a lot simpler than the same thing in C++,
because we can use Python's dictionary type for data storage. SCOP
doesn't support dictionaries directly though, so we need marshalling
functions to convert it to something it does understand (lists of
lists, in this case). In fact this is so useful I might add
dictionary support to the library at some point, although then this
example would be rather pointless!

\codebar
\small
\begin{verbatim}
# xml_sender.py - DMI - 11-11-03

import scop

def marshall(dict):
   l = []
   keys = dict.keys()
   for k in keys:
      l.append([k, dict[k]])
   return l

ab = { "Poirot": "Belgium",
       "Morse": "Oxford, UK",
       "Danger Mouse": "London, UK" }
v = marshall(ab)
sock = scop.scop_open("localhost", "xml_sender")
scop.scop_send_struct(sock, "xml_receiver", v)
scop.scop_close(sock)
\end{verbatim}
\normalsize
\codebar
\small
\begin{verbatim}
# xml_reciever.py - DMI - 11-11-03
# Usage: xml_reciever [-inspect]

import scop, sys
from scopxml import pretty_print

def extract(v):
   d = {}
   for pair in v:
      key, value = pair
      d[key] = value
   return d

sock = scop.scop_open("localhost", "xml_receiver")
v, rpc_flag = scop.scop_get_struct(sock)
if len(sys.argv) == 2 and sys.argv[1] == "-inspect":
   s = pretty_print(v)
   print s
ab = extract(v)
print ab
scop.scop_close(sock)
\end{verbatim}
\normalsize
\codebar

\subsection{XML RPC functions}

\code{%
scop\_rpc(sock, endpoint, request, method = None)\\
\ret{Returns polymorphic, or None on error}\\
scop\_send\_reply(sock, reply)
\ret{No return value}\\
scop\_get\_request(sock)
\ret{Returns polymorphic or None if not an incoming RPC}
}

Notice that the same \texttt{scop\_rpc()} and
\texttt{scop\_send\_reply()} functions are used as for
non-XML RPC's. These functions perform different operations based on the types
of their \texttt{request} and \texttt{reply} arguments. If these are
plain strings then XML is not used, otherwise it is.

A subtle consequence of this is
that it is not possible to send an XML message containing a single
value of type string using the Python binding. Fortunately there is no
particularly good reason to do this.

\subsubsection*{XML RPC example}

\codebar
\small
\begin{verbatim}
# xml_client.py - DMI - 11-11-03
# Usage: xml_client [<n> <k>]    (default values n = 4, k = 2)

import scop, sys

if len(sys.argv) != 3:
   n, k = 4, 2
else:
   n, k = int(sys.argv[1]), int(sys.argv[2])
sock = scop.scop_open("localhost", "xml_client")
v = [n, k]
w = scop.scop_rpc(sock, "xml_server", v)
print str(n) + " choose " + str(k) + " equals " + str(w) + "."
scop.scop_close(sock)
\end{verbatim}
\normalsize
\codebar
\small
\begin{verbatim}
# xml_server.cpp - DMI - 11-11-03

import scop

def combi(n, k):
   result = 1;
   if k > n or k < 0:
      return 0 
   for i in range(k):
      result *= n - i
   for i in range (1, k + 1):
      result /= i
   return result

sock = scop.scop_open("localhost", "xml_server")
while 1:
   v = scop.scop_get_request(sock)
   w = combi(v[0], v[1])
   scop.scop_send_reply(sock, w)
scop.scop_close(sock)
\end{verbatim}
\normalsize
\codebar

\subsection{XML RPC method names}

There are no \texttt{extract\_method()} and \texttt{extract\_args()}
functions in the Python binding, since these can be achieved
simply by assigning the returned object to a tuple.

\subsubsection*{Multiple RPC methods example}

\codebar
\small
\begin{verbatim}
# method_client.py - DMI - 11-11-03

import scop

def cent_to_faren(sock, c):
   return scop.scop_rpc(sock, "method_server", c, "ctof")

def faren_to_cent(sock, f):
   return scop.scop_rpc(sock, "method_server", f, "ftoc")

def count_uses(sock):
   return scop.scop_rpc(sock, "method_server", None, "stats")

sock = scop.scop_open("localhost", "method_client")
print str(0.0) + " deg C = " + str(cent_to_faren(sock, 0.0)) + " deg F."
print str(20.0) + " deg C = " + str(cent_to_faren(sock, 20.0)) + " deg F."
print str(60.0) + " deg F = " + str(faren_to_cent(sock, 60.0)) + " deg C."
print "The server has been accessed " + str(count_uses(sock)) + " times."
scop.scop_close(sock)
\end{verbatim}
\normalsize
\codebar
\small
\begin{verbatim}
# method_server.py - DMI - 11-11-03

import scop, sys

invocations = 0

def cent_to_faren(c):
   global invocations
   invocations += 1
   return (9.0 * c / 5.0) + 32.0

def faren_to_cent(f):
   global invocations
   invocations += 1
   return (f - 32.0) * 5.0 / 9.0

sock = scop.scop_open("localhost", "method_server")
while 1:
   v = scop.scop_get_request(sock);
   method, args = v
   if method == "ctof":
      w = cent_to_faren(args)
   elif method == "ftoc":
      w = faren_to_cent(args)
   elif method == "stats":
      w = invocations
   else:
      sys.exit()
   scop.scop_send_reply(sock, w)
scop.scop_close(sock)
\end{verbatim}
\normalsize
\codebar

\subsection{XML Cookies}

\code{%
scop\_get\_cookie(sock, name)
\ret{Returns polymorphic or None}\\
scop\_set\_cookie(sock, data)
\ret{No return value}
}

\subsection{Debugging}

These internal functions are defined in the module \texttt{scopxml},
and may occasionally be helpful if you want to access the XML
parser directly for some reason.

\code{%
vertex\_to\_string(v, method = None)
\ret{Returns a string}\\
string\_to\_vertex(s)
\ret{Returns polymorphic}\\
pretty\_print(v)
\ret{Returns a string}
}

\appendix
\section{Performance}

The local-case performance was measured on an 800 Mhz Pentium III as
890 RPC's per second. This is slightly slower than the Java binding
(1000 round trips per second) and 7.5 times slower than C++ (6700 per
second). Of course the \texttt{scopserver} process is implemented
in C++ either way.

It's interesting to note the effect on code size using Python: in C++,
the main library is 782 lines and the XML conversion routines are 963
lines. In Python, the library is only 332 lines and the XML routines
are 144 lines.

\end{document}
