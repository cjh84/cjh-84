\documentclass[12pt,a4paper,twoside]{article}
\usepackage{fancyhdr}

\setlength{\parindent}{0cm}
\setlength{\parskip}{2ex plus1ex minus 0.5ex}

\addtolength{\evensidemargin}{-2.5cm}
\addtolength{\oddsidemargin}{-0.5cm}
\addtolength{\textwidth}{3cm}

\addtolength{\headheight}{0.2cm}
\addtolength{\topmargin}{-1cm}
\addtolength{\textheight}{2.5cm}
% \addtolength{\footskip}{0.5cm}

\renewcommand{\_}{\texttt{\symbol{95}}}
\addtolength{\fboxsep}{0.1cm}
\newcommand{\param}[1]{\textit{\textrm{\textmd{#1}}}}
\newcommand{\codebar}{\rule{\textwidth}{0.3mm}}

\newlength{\codelen}
\newcommand{\code}[1]
{\begin{center}\fbox{\parbox{16cm}{\texttt{#1}}}\end{center}}

\fancyhead{}
\fancyhead[RO,LE]{\thepage}
\fancyhead[LO,RE]{SCOP Overview and Installation Guide}
\fancyfoot{}
\pagestyle{fancy}

\input{macros}
\begin{document}

\centerline{\textbf{\LARGE SCOP Overview and Installation Guide}}
\vspace{0.5cm}
\centerline{Version 1.5, 11th September 2006}
\centerline{David Ingram (\texttt{dmi1000@cam.ac.uk})}

{ \parskip 1mm plus 1pt \tableofcontents }

\section{Overview}

\subsection{What does it do?}

SCOP consists of a user library and server process designed to make
writing distributed applications (particularly in C, C++, Python and
Java) very easy. It allows different processes (which may be written in
different languages, on different machines and architectures --- or the
same machine of course) to communicate by means of sending each other
\textit{messages} or \textit{events}, with very little effort from the
programmer.

\subsection{Why is it better than XYZ?}

UNIX/Linux lacks a standard, built-in events system and this has been
detrimental to the development of large scale integrated projects such
as desktop environments and component embedding. Pipes and scripts do a
good job for largely sequential command-line utilities; but there is no
easy way to apply the UNIX philosophy of sharing data, automating and
combining smaller tools to concurrent, possibly distributed, graphical
programs.

There are plenty of existing events systems which you can install, but
these tend to be either (i) huge and clunky, such as CORBA (frustration
with which motivated this project), (ii) unpolished research projects,
or (iii) special-purpose, for example almost every window system and
graphics toolkit has an internal events system. SCOP is small, complete
and general-purpose. The acronym stands rather arbitrarily for Systems
COmmunication Protocol (by analogy with KDE's DCOP, which does
something similar in the context of a desktop environment). It is
pronounced ``es-cop''.

\subsection{Features}

The functionality offered by SCOP includes:

\begin{bulletlist}
\item Message passing
\item Events
\item Resource discovery
\item RPC
\end{bulletlist}

Of course message passing and event delivery can be handled by the same
mechanism, and indeed I shall make no further distinction between
them. Resource discovery is facilitated by the ability to query which
clients are connected to the server, and retrieve some state (a cookie)
associated with each active connection. There are no problems with
state left over from dead connections, unlike some file locking
solutions.

RPC (Remote Procedure Call) is supported to the extent of
allowing messages to have replies, and providing separate XML-based
routines to package up (marshal) arguments and results. There is quite
deliberately no IDL or automatic stub generation, so no
preprocessing or machine-generated code is involved. I feel that the
advantages of understanding exactly what is going on and not having a
special build process imposed upon your project outweighs the alleged
disadvantage of remote function calls not looking exactly like local
ones. The extra work required to manually wrap up arguments is quite
modest, and forces the programmer to think about efficiency a little
bit whilst they are doing it. Although the system does not therefore
provide \textit{distribution transparency}, it is in another sense
quite transparent by making it obvious to the reader of the code
exactly when, where and what messages are being sent over the network
and which format they are in.

\subsection{Good things}

Here are some good things about SCOP:

\begin{bulletlist}
\item Very simple to learn and use.
\item Cleanly layered APIs allow you to use more rudimentary low-level
interfaces if you don't want extra fancy high level features.
\item Plain text wire format (with a basic XML subset if you are
using the typed API), so if you don't know what is going on you
can always inspect the messages being sent, and parse them yourself
if you like.
\item Memory management nightmares with C/C++ are avoided (without
introducing error-prone pseudo-garbage-collection features)
through a clearly defined and straightforward allocation/deallocation
rule.
\item Integrates with other file and socket based I/O through
the standard \texttt{select} call.
\item Doesn't take over your thread of control by imposing an
event loop.
\item No additional threads created (unless you do so explicitly
yourself).
\end{bulletlist}

Don't be put off by the length of this manual, because simple
things are still very, very simple to do. For example, here's how
to multicast a message to a bunch of other processes:

\begin{verbatim}
int sock = scop_open("localhost", "myclient");
scop_send_message(sock, "groupname", "Hello, world!");
\end{verbatim}

Fantastic \texttt{:-)}

\subsection{Performance}

SCOP is very lightweight, so although speed wasn't the main design
goal, performance is reasonably good. My intuition is that the overhead
due to the use of ASCII messages compares pretty well with the complex
type system processing incorporated in most binary alternatives.

Typical figures for same-machine communication are a round trip
time of 150$\mu$s on a 800 Mhz Pentium III (around 6700 sequential
RPC's per second). Inter-machine communication was measured as 450$\mu$s
per round trip, giving 2300 sequential RPC's per second.

\subsection{Bad things}

A current weakness of SCOP is that it has rather primitive security, and
very little error checking. It's easy to crash the clients through
incorrect use of the protocol, if you want to. It's almost certainly
possible to crash the server likewise, although the design is supposed
to make this less likely. The server is single-threaded so denial of
service attacks are trivial. Of course, the source code is available so
anyone can strengthen the security if this becomes necessary. Otherwise
it is advisable to run it inside a firewall.

The two security features which \textit{are} provided are privilege
reduction and hostname access control. If you run \texttt{scopserver}
as root it will automatically change itself to user and group
\texttt{nobody} (assuming they exist on your system) as soon as it
starts, which helps by limiting the mischief a runaway server could do
quite a bit. \texttt{scopserver} will only allow access from clients
running on hosts which are listed in its authorised host file. This
means that any user on any permitted machine has unconstrained access
to SCOP, but hosts you haven't explicitly included cannot connect at
all.

\section{Getting started}
\label{installation}

\subsection{Requirements}

Nothing specific, other than \texttt{gcc} and GNU \texttt{make}.

SCOP has been developed on Linux, but should be portable to UNIX
environments with a little effort. It has been tested successfully on
Slackware, Mandrake and Redhat Linux systems, on FreeBSD and on Solaris.
I would be interested to hear what changes need to be made to get it to
build on other UNIX variants.

\subsection{Where to get SCOP}

You can download it from

\verb=http://www.srcf.ucam.org/~dmi1000/scop/=

\subsection{Compilation}

\begin{verbatim}
tar xzvf scop.tar.gz
cd scop
make
\end{verbatim}

\subsection{Installation as root}

If you have root access on your chosen machine, you should
install SCOP like this. It is perfectly possible to install
it without being root, but slightly less convenient in use.

\begin{verbatim}
su -c make install
su -c echo >>/etc/services "scop 51234/tcp"
\end{verbatim}

Note: you can choose a different port number in the last line if you
like, as long as you pick the same one for each machine which needs to
communicate. The services entry must exist even on machines which
only run clients and don't need the server process installed.

\subsection{Non-root installation}

If you don't have root access to the machine, SCOP can be
installed like this.

\begin{verbatim}
mkdir ~/include ~/lib ~/bin
make userinstall
\end{verbatim}

You should also add the following to your shell startup script
(\texttt{.bashrc}, for example):

\begin{verbatim}
export PATH=$PATH:~/bin
export SCOP_LOGFILE=~/.scoplog
\end{verbatim}

Since you can't update \texttt{/etc/services}, scop will use the
standard port number (51234) when it finds it can't look up the service
entry. This value can be changed at compile time by editing the
constant \texttt{FALLBACK\_PORT} in \texttt{scop.h}

\subsection{Configuration}

The only configuration task is to define the set of machines which
should be allowed to connect to SCOP. This is done by editing the host
access control file, \texttt{/etc/scophosts} (root installations) or
\verb=~/.scophosts= (non-root installations). The format consists of
fully-qualified host names, e.g. \texttt{machinename.foo.bar.com} or
whatever, one per line.

The default file just contains the special name \texttt{localhost},
which allows connections from the same machine that you are going to
run the \texttt{scopserver} process on. If you have trouble connecting
to the server you might need to add the name of your machine explicitly
(not doing this is a common cause of mysterious failure). The SCOP
server process only reads this file when it starts up, so you must kill
and then restart the server if you subsequently modify the list of
authorised hosts.

\subsection{Starting the server}
\label{starting}

If necessary add \texttt{/usr/local/sbin} to your \verb=$PATH=,
then simply run

\texttt{scopserver}

In a non-root installation, run this instead:

\verb=scopserver -f ~/.scophosts=

The server will detach and run in the background automatically. You
probably want to add this command to the system's
\texttt{/etc/rc.d/rc.local} file so the server starts automatically at
boot time, for root installations. It is not possible to run more than
one copy of the server per machine. In fact neither message sources nor
sinks need a locally running server, since they can be directed to
route via a server running on a third machine.

Check the server is running by looking for \texttt{scopserver} in the
system's process list. If it doesn't seem to have started correctly,
you should look for error messages in the SCOP log (see the next
section). All problems are reported there, and not to the console.

\subsubsection*{Log file location}

If you have set the \texttt{SCOP\_LOGFILE} environment variable,
log messages are appended to the file you specified.

If \texttt{SCOP\_LOGFILE} isn't set, an \texttt{scopserver} running as
root will use the \texttt{syslog} facility to issue log messages
instead. They will therefore generally appear in a system log file such
as \texttt{/var/log/messages}. You can change \texttt{/etc/syslog.conf}
if this isn't the case.

In cases where \texttt{SCOP\_LOGFILE} isn't set but \texttt{scopserver}
is running as a normal user (not root), messages are logged to a
default location which is \texttt{~/.scoplog} instead. The reason we
don't use \texttt{syslog} for non-root installations is that it
typically isn't readable by ordinary users. Don't forget you can
override all this by setting \texttt{SCOP\_LOGFILE} yourself (perhaps
to \texttt{/tmp/scoplog}, for example).

\subsubsection*{Server options}

As well as the use of \texttt{scopserver -f <hostfile>} to specify
the location of the access-control file, there are two other options
you can pass to \texttt{scopserver}:

\texttt{scopserver --allhosts} deactivates access control altogether,
allowing incoming connections from any machine.

\texttt{scopserver -p <port-number>} tells the server to listen
for incoming connections on the port specified. This overrides
any \texttt{/etc/services} entry and the default port number.

\subsection{Testing it}
\label{testing}

Once the server is running, you can easily test your installation using
the command line tools \texttt{scop} and \texttt{listen}, which will
also have been installed. These clients should be run as a normal
user; they don't require any special privilege. You are encouraged
to work through the following tests since they also help illustrate
SCOP's capabilities.

If you get ``\texttt{Broken pipe}'' error messages or
things don't seem to be working, look at the SCOP log file, as
described in \S\ref{starting}. Any problems discovered in the SCOP
library cause explanatory messages to be logged there, so it provides a
single place to look for errors both from \texttt{scopserver} and the
client programs.

Incidentally, you can instruct the server to be more verbose
in its logging (typically this means logging successful commands
as well as errors) by issuing the command \texttt{scop log 1}.
You need to do this after \texttt{scopserver} has started,
but before the behaviour you wish to monitor. It will generate a
lot of output if your system is handling a high rate of events.
A return to normal log verbosity is achieved with \texttt{scop log 0}.

For a quick and effective demonstration of multiple endpoints, you
should open several shell windows on the machine which is running the
server; lets say five windows for this test. Resize and shuffle them
around until you can see all of them. In this section commands to be
typed are prefixed by the window number into which you should type
them. Then proceed as follows (don't be concerned if some of the
commands appear to pause indefinitely -- they are waiting for events):

\begin{verbatim}
1: scop --help
1: scop list
2: listen --help
2: listen magic
3: listen spell
4: listen magic
1: scop list
1: scop send spell "hello world"
1: scop send magic "foo"
1: scop send rhubarb "foo"
1: scop verify magic "bar"
1: scop verify rhubarb "bar"
1: scop rpc spell "backwards"
1: ping localhost | stream magic
5: scop list
5: listen -u magic
1: CTRL-C
1: scop list
1: scop clear spell
1: scop list
5: CTRL-C
1: scop list
\end{verbatim}

Pay attention to the output you see in each window; it should be
self-explanatory. The communication endpoints we set up in this example
are called ``magic'' and ``spell''. Note that multicast is performed
automatically if several clients connect with the same name, and
messages sent to non-existant endpoints are silently dropped. The
\texttt{ping} command is used as an example stream source because on
most systems it will generate a new line of output every second.

\subsection{Networked usage}

When you wish to communicate between several different machines,
you should only run \texttt{scopserver} on one of them
(separate instances of \texttt{scopserver} are separate
worlds with their own endpoint namespaces, which don't talk
to each other).

You must tell the clients which are running on the other machines
where the \texttt{scopserver} is located. You can do this with
the \texttt{-r <hostname>} option to \texttt{scop} and \texttt{listen}.
If you miss out the \texttt{-r} argument it is as if you said
\texttt{-r localhost}, which is the standard alias expanding
to the name of the machine the process is running on.

\subsection{Linking your application with SCOP}

Shared library support is being added to future versions, but
currently the library has to be statically linked (fortunately it is
rather small). Here's how to do it.

In your source code, include the line

\begin{verbatim}
#include <scop.h>
\end{verbatim}

Then build your application with a Makefile like this one:
\begin{verbatim}
foo: foo.cpp
   g++ -o foo foo.cpp -lscop
\end{verbatim}

What could be easier? \texttt{:-)}

If SCOP has \textit{not} been installed as root you need some extra flags,
so use a Makefile like this instead:
\begin{verbatim}
foo: foo.cpp
   g++ -I ${HOME}/include -L ${HOME}/lib -o foo foo.cpp -lscop
\end{verbatim}

\section{API Level 4 - Command Line Interface}

SCOP provides several layers of API, the highest of which is the
command line interface (``layer 4''). Here is a brief summary of
the shell commands available, for reference.

\subsection{listen}

\begin{verbatim}
Usage: listen [option...] <interest>

Options: -r <hostname>   remote scopserver
         -u              unique endpoint
         -p              persistent
         -e <command>    execute command with message arguments
\end{verbatim}

The \texttt{listen} program displays the messages sent to an
endpoint which you specify on the command line.
For the sake of example it tries to guess if they are in XML
format (pretty-printing them if so), and illustrates
replying to RPCs by reversing the argument, treated as a
string. More realistic servers would of course know what
format messages to expect!

The \texttt{-e} option is intended to support quick and dirty
remote-control functionality via shell scripts (not recommended
for serious work --- use the C API instead!) If you run

\verb=listen -e "shell command" <endpoint>=

Then whenever a message with content \verb=foo bar= is sent to
\texttt{<endpoint>}, listen will execute the following as if
you had typed it into the shell:

\verb=$ shell command foo bar=

The \texttt{-p} option allows \texttt{listen} to continue running if the
\texttt{scopserver} it is connected to exits. The \texttt{listen} process will
then periodically attempt to reconnect to \texttt{scopserver}. It can also wait
for \texttt{scopserver} to start if it isn't running initially.

The algorithm at present tries reconnecting once per second for a
minute, then once every 2 seconds for 2 minutes, then once every
4 seconds for 4 minutes and so on. After trying once every 32
seconds for 32 minutes the program retries every 64 seconds indefinitely.

This feature together with the \texttt{-p} option for \texttt{stream}, below,
enable one to create very reliable streams which automatically restart without
failing when the \texttt{scopserver} process is killed (or, more probably,
someone reboots the machine it is on). If you have cron jobs on the respective
machines to restart the \texttt{scopserver}, the \texttt{stream} source and the
\texttt{listen} process then you know everything will correctly restart itself
regardless of the type of failure experienced.

\subsection{stream}

\begin{verbatim}
Usage: stream [option...] <endpoint>
Options: -r <hostname>   remote scopserver
         -p              persistent
\end{verbatim}

The \texttt{stream} program reads lines of text from standard input,
and sends each one in turn as a message to the endpoint specified.
Piping the output of another command into \texttt{stream}
is a very quick and easy way of converting an external event source into
SCOP events.

The \texttt{-p} option works the same way as for the \texttt{listen} client,
above. Whilst \texttt{scopserver} is down, \texttt{stream} will still keep
reading from \texttt{stdin} to prevent a stall in the pipeline, but it will
discard the incoming messages until \texttt{scopserver} reappears.

Note that a more sophisticated system would probably offer the option of
buffering the messages and replaying them after \texttt{scopserver} is
restarted in order not to lose any information. We can't do this in a
straightforward way because it's unlikely the listeners would have restarted as
soon as the \texttt{scopserver} is back, hence the replayed messages would
probably be dropped by \texttt{scopserver} for want of a receiver anyway. A
correct implementation of buffered messages would also have to execute
\texttt{scop query} on the endpoint in question until the required listener[s]
were back, before initiating message replay.

\subsection{scop}

\begin{verbatim}
Usage: scop [<opts>] send {<endpoint> <message>}+
       scop [<opts>] verify <endpoint> <message>
       scop [<opts>] list
       scop [<opts>] log 0|1
       scop [<opts>] query <endpoint>
       scop [<opts>] clear <endpoint>
       scop [<opts>] rpc <name> <message>
       scop [<opts>] xmlrpc <name> <method or "-"> <arg>+
       scop [<opts>] xmlsend <name> <method or "-"> <arg>+
       scop [<opts>] terminate
       scop [<opts>] reconfigure

Options: -r <hostname>
\end{verbatim}

\texttt{scop} is a general-purpose command line client. It is
mainly used for sending messages and listing the active
connections. See the walkthrough in \S \ref{testing} for some
examples of its use.

\appendix
\section{License notice}

Copyright (C) David Ingram, 2001-02, \texttt{dmi1000@cam.ac.uk}.

This library is free software; you can redistribute it and/or
modify it under the terms of the GNU Lesser General Public
License as published by the Free Software Foundation; either
version 2.1 of the License, or (at your option) any later version.

This library is distributed in the hope that it will be useful,
but WITHOUT ANY WARRANTY; without even the implied warranty of
MERCHANTABILITY or FITNESS FOR A PARTICULAR PURPOSE.  See the GNU
Lesser General Public License for more details.

You should have received a copy of the GNU Lesser General Public
License along with this library; if not, write to the Free Software
Foundation, Inc., 59 Temple Place, Suite 330, Boston, MA  02111-1307  USA

\end{document}
